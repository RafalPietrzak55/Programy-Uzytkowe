\documentclass[12pt, letterpaper, titlepage]{article}
\usepackage[left=3.5cm, right=2.5cm, top=2.5cm, bottom=2.5cm]{geometry}
\usepackage[MeX]{polski}
\usepackage[utf8]{inputenc}
\usepackage{graphicx}
\usepackage{enumerate}
\usepackage{amsmath} %pakiet matematyczny
\usepackage{amssymb} %pakiet dodatkowych symboli
\usepackage{multirow}
\usepackage{xcolor}
\title{ćw3 latex}
\author{Rafał Pietrzak}
\date{Listopad 2022}
\begin{document}
\maketitle

\section{.zadanie}

\begin{table}[h]
\centering\caption{Przykładowy system decyzyjny (U,A,d), modelujący problem diagnozy medycznej, której efektem jest decyzja o wykonaniu lub nie wykonaniu operacji wycięcia wyrostka robaczkowego, U = \{$u_{1},u_{2},....,u_{10}$\} ,A = \{$a_{1},a_{2}$\},$d\in D$ =\{TAK,NIE\}}
\begin{tabular}{c|c c c}
\hline
\hline
Pacjent & Ból brzucha & Temperatura ciała & Operacja\\
\hline
u1 & Mocny & Wysoka & \textcolor{green}{Tak}\\

u2 & Średni & Wysoka & \textcolor{green}{Tak}\\

u3 & Mocny & Średnia & \textcolor{green}{Tak}\\

u4 & Mocny & Niska & \textcolor{green}{Tak}\\

u5 & Średni & Średnia &  \textcolor{green}{Tak}\\

u6 & Średni & Średnia & \textcolor{red}{Nie} \\

u7 & Mały & Wysoka & \textcolor{red}{Nie}\\

u8 & Mały & Niska & \textcolor{red}{Nie}\\

u9 & Mocny & Niska & \textcolor{red}{Nie}\\

u10 & Mały & Średnia & \textcolor{red}{Nie}\\
\hline
\hline

\end{tabular}
\end{table}


\newpage
\section{.zadanie}

\begin{table}[h]
\centering\caption{Bramka Logiczna AND}
\begin{tabular}{c|c|c}

p & q & s\\
\hline
\hline
0 & 0 & 0\\
\hline
0 & 1 & 0\\
\hline
1 & 0 & 0\\
\hline
1 & 1 & 1\\

\end{tabular}
\end{table}

\begin{table}[h]
\centering\caption{Bramka Logiczna NOT}
\begin{tabular}{c|c|c}

p & q & s\\
\hline
\hline
0 &  & 1\\
\hline
1 &  & 0\\


\end{tabular}
\end{table}

\begin{table}[h]
\centering\caption{Bramka Logiczna NAND}
\begin{tabular}{c|c|c}

p & q & s\\
\hline
\hline
0 & 0 & 1\\
\hline
0 & 1 & 1\\
\hline
1 & 0 & 1\\
\hline
1 & 1 & 0\\

\end{tabular}
\end{table}

\begin{table}[h]
\centering\caption{Bramka Logiczna OR}
\begin{tabular}{c|c|c}

p & q & s\\
\hline
\hline
0 & 0 & 0\\
\hline
0 & 1 & 1\\
\hline
1 & 0 & 1\\
\hline
1 & 1 & 1\\

\end{tabular}
\end{table}

\begin{table}[h]
\centering\caption{Bramka Logiczna OR}
\begin{tabular}{c|c|c}

p & q & s\\
\hline
\hline
0 & 0 & 0\\
\hline
0 & 1 & 1\\
\hline
1 & 0 & 1\\
\hline
1 & 1 & 1\\

\end{tabular}
\end{table}

\begin{table}[h]
\centering\caption{Bramka Logiczna NOR}
\begin{tabular}{c|c|c}

p & q & s\\
\hline
\hline
0 & 0 & 1\\
\hline
0 & 1 & 0\\
\hline
1 & 0 & 0\\
\hline
1 & 1 & 0\\

\end{tabular}
\end{table}

\begin{table}[h]
\centering\caption{Bramka Logiczna XOR}
\begin{tabular}{c|c|c}

p & q & s\\
\hline
\hline
0 & 0 & 0\\
\hline
0 & 1 & 1\\
\hline
1 & 0 & 1\\
\hline
1 & 1 & 0\\

\end{tabular}
\end{table}

\end{document}